% !Mode:: "TeX:UTF-8"
\documentclass{beamer}

\usetheme{Darmstadt}
\useinnertheme{rounded}

\usecolortheme{beaver}
%\usecolortheme{albatross}
%\usecolortheme{beetle}
%\usecolortheme{crane}
%\usecolortheme{dolphin}
%\usecolortheme{dove}
%\usecolortheme{fly}
%\usecolortheme{lily}
%\usecolortheme{orchid}
%\usecolortheme{rose}
%\usecolortheme{seagull}
%\usecolortheme{seahorse}
%\usecolortheme{whale}
%\usecolortheme{wolverine}
%\usecolortheme{default}

\numberwithin{subsection}{section}
\usefonttheme[onlylarge]{structurebold}
\setbeamercovered{transparent}

\setbeamerfont*{frametitle}{size=\normalsize,series=\bfseries}
\setbeamertemplate{navigation symbols}{}

\input{en_preamble.tex}
\input{xecjk_preamble.tex}

\title{《数值代数》研究生精品课程建设项目中期汇报}
\author{汇报人:魏华祎}

\AtBeginSection[]
{
  \frame<beamer>{ 
    \frametitle{Outline}   
    \tableofcontents[currentsection] 
  }
}

\AtBeginSubsection[]
{
  \frame<beamer>{ 
    \frametitle{Outline}   
    \tableofcontents[currentsection] 
  }
}

\begin{document}

\begin{frame}
  \titlepage
\end{frame}

\begin{frame}{Outline}
  \tableofcontents
\end{frame}

\section{课程介绍}
\begin{frame}{课程介绍}
    科学与工程计算中的大多数问题经过数值离散,最后都归结为矩阵计算问题,因此矩
    阵计算是科学与工程计算的核心,是计算数学各专业都必须掌握的基础知识。本课程
    主要介绍如何结合各类科学与工程问题中所得到的矩阵计算问题的特点,设计出相应
    快速可靠算法的一些基本知识和方法。
\end{frame}

\section{教师团队}
\begin{frame}{教师团队}
    目前教师团队中有一名教授,两名副教授,且都是活跃在科研一线的青年骨干教师,积
    累一丰富的科研成果和经验。
    \begin{itemize}
        \item[$\bullet$] 魏华祎(副教授),项目负责人
        \item[$\bullet$] 易年余(教授,副院长)
        \item[$\bullet$] 陆汛(副教授)
    \end{itemize}
\end{frame}

\begin{frame}{魏华祎}
\only<1>{
    学习工作经历:
    \begin{itemize}
        \item[(1)] 2015/01-至今, 湘潭大学, 数学与计算科学学院, 副教授.
        \item[(2)] 2012/07-2014/12, 湘潭大学, 数学与计算科学学院, 讲师.
        \item[(3)] 2010/12–2011/12, 美国加州大学欧文分校,计算数学,联合培养博
            士生,导师: 陈龙.
        \item[(4)] 2008/09–2012/06, 湘潭大学, 计算数学, 博士, 导师: 黄云清.
        \item[(5)] 2006/09–2008/07, 湘潭大学, 计算数学, 硕士生, 导师: 陈艳萍.
        \item[(6)] 2002/09–2006/07, 河南大学, 数学与应用数学, 学士.
    \end{itemize}
}
\only<2>{
    研究方向:
    \begin{itemize}
        \item[$\bullet$] 高效数值模拟软件研制
        \item[$\bullet$] 高质量网格生成优化方法及应用
        \item[$\bullet$] 自适应虚单元方法理论及应用
    \end{itemize}
    项目负责人在SIAM、JCP 等高水平期刊发表论文十余篇,主持国家自科项目
    2 项,横向项目 1 项,参与国家自科重大研究计划 1 项及多项国防军工项目。目前
    已指导 9 名研究生,其中 3 名顺利毕业,1 名转硕博连读。
}
\end{frame}

\begin{frame}{易年余}
\only<1>{
    学习工作经历:
    \begin{itemize}
        \item[(1)] 
        \item[(2)] 
        \item[(3)] 
        \item[(4)]
        \item[(5)]
        \item[(6)]
    \end{itemize}
}
\only<2>{
    研究方向:
    \begin{itemize}
        \item[$\bullet$] 有限元超收敛与重构技术
        \item[$\bullet$] 后验误差估计与自适应算法 
        \item[$\bullet$] 间断有限元方法
        \item[$\bullet$] 保持偏微分方程特征的数值方法
        \item[$\bullet$] 最优控制问题数值方法
    \end{itemize}
}
\end{frame}

\begin{frame}{陆汛}
\only<1>{
    学习工作经历:
    \begin{itemize}
        \item[(1)] 
        \item[(2)] 
        \item[(3)] 
        \item[(4)]
        \item[(5)]
        \item[(6)]
    \end{itemize}
}
\only<2>{
    研究方向:
    \begin{itemize}
        \item[$\bullet$] 计算光子学
        \item[$\bullet$] Maxwell方程数值解法
    \end{itemize}
}
\end{frame}
\section{目前建设情况}
\begin{frame}{课件}
\end{frame}

\begin{frame}{课程中心}
\end{frame}

\begin{frame}{课程网站}
\end{frame}

\begin{frame}{协作阅读平台}
\end{frame}

\begin{frame}{软件平台}
\end{frame}

\section{总结}
\begin{frame}{总结}
\end{frame}
\section{个人简历}


\end{document}
