
\documentclass[notheorems,serif]{beamer}

%选用主题
%\usetheme{Rochester}
%\usetheme{default}
%\usetheme{AnnArbor}
%\usetheme{Antibes}
%\usetheme{Bergen}
%\usetheme{Berkeley}
%\usetheme{Berlin}
%\usetheme{Boadilla}
%\usetheme{CambridgeUS}
%\usetheme{Copenhagen}
%\usetheme{Darmstadt}
%\usetheme{Dresden}
%\usetheme{Frankfurt}
%\usetheme{Goettingen}%
%\usetheme{Hannover}
%\usetheme{Ilmenau}
%\usetheme{JuanLesPins}
%\usetheme{Luebeck}
\usetheme{Madrid}
%\usetheme{Malmoe}
%\usetheme{Marburg}
%\usetheme{Montpellier}
%\usetheme{PaloAlto}
%\usetheme{Pittsburgh}
%\usetheme{Rochester}
%\usetheme{Singapore}
%\usetheme{Szeged}
%\usetheme{Warsaw}

% As well as themes, the Beamer class has a number of color themes
% for any slide theme. Uncomment each of these in turn to see how it
% changes the colors of your current slide theme.

%\usecolortheme{albatross}
%\usecolortheme{beaver}
%\usecolortheme{beetle}
%\usecolortheme{crane}
%\usecolortheme{dolphin}
%\usecolortheme{dove}
%\usecolortheme{fly}
%\usecolortheme{lily}
%\usecolortheme{orchid}
%\usecolortheme{rose}
%\usecolortheme{seagull}
%\usecolortheme{seahorse}
%\usecolortheme{whale}
%\usecolortheme{wolverine}

%设置被cover的内容不显示
%\setbeamercovered{transparent}

\useinnertheme{rounded}
\usecolortheme{default}

%调用包
\usepackage[no-math, cm-default]{fontspec}
\usepackage{xltxtra}
\usepackage{xunicode}   
\usepackage{xcolor}
\usepackage{amsmath,amssymb}
\usepackage{xeCJK}
\usepackage{multimedia}
\usepackage{listings}
\usepackage{subfigure}
\usepackage{todonotes}
\presetkeys{todonotes}{inline}{} 
\usepackage{multicol}
\usepackage{changes}




%将系统字体名映射为逻辑字体名称,主要是为了维护的方便  
\newcommand\fnhei{Adobe 黑体 Std}  
\newcommand\fnsong{Adobe 宋体 Std}  
\newcommand\fnkai{Adobe 楷体 Std}  
\newcommand\fnmono{DejaVu Sans Mono}  
\newcommand\fnroman{Times New Roman}  

\renewcommand{\normalsize}{\wuhao}

%%设置常用中文字号,方便调用  
\newcommand{\erhao}{\fontsize{22pt}{\baselineskip}\selectfont}  
\newcommand{\xiaoerhao}{\fontsize{18pt}{\baselineskip}\selectfont}  
\newcommand{\sanhao}{\fontsize{16pt}{\baselineskip}\selectfont}  
\newcommand{\xiaosanhao}{\fontsize{15pt}{\baselineskip}\selectfont}  
\newcommand{\sihao}{\fontsize{14pt}{\baselineskip}\selectfont}  
\newcommand{\xiaosihao}{\fontsize{12pt}{\baselineskip}\selectfont}  
\newcommand{\wuhao}{\fontsize{10.5pt}{\baselineskip}\selectfont}  
\newcommand{\xiaowuhao}{\fontsize{9pt}{\baselineskip}\selectfont}  
\newcommand{\liuhao}{\fontsize{7.5pt}{\baselineskip}\selectfont}  

%\setmainfont{\fnroman}
\setmainfont{\fnkai}
\setCJKmainfont[BoldFont=\fnhei]{\fnkai}  
\setCJKsansfont[BoldFont=\fnhei]{\fnkai}  
\setCJKmonofont{\fnkai}  

%楷体  
%\newfontinstance\KAI{\fnkai}  
%\newcommand{\kai}[1]{{\KAI#1}}  
%黑体  
%\newfontinstance\HEI{\fnhei}  
%\newcommand{\hei}[1]{{\HEI#1}}  
%英文  
%\newfontinstance\ENF{\fnroman}  
%\newcommand{\en}[1]{\,{\ENF#1}\,}

%楷体  
\newfontfamily\KAI {\fnkai}  
\newcommand{\kai}[1]{{\KAI#1}}  
%黑体  
\newfontfamily\HEI{\fnhei}  
\newcommand{\hei}[1]{{\HEI#1}}  
%英文  
\newfontfamily\ENF{\fnroman}  
\newcommand{\en}[1]{\,{\ENF#1}\,}


%连字符
\defaultfontfeatures{Mapping=tex-text}

%中文断行
\XeTeXlinebreaklocale "zh"
\XeTeXlinebreakskip = 0pt plus 1pt minus 0.1pt


%%%% 定理类环境的定义 %%%%
\newtheorem{example}{\hei{例子}} 
\newtheorem{problem}{\hei{问题}}           
\newtheorem{algorithm}{\hei{算法}}
\newtheorem{theorem}{\hei{定理}}
\newtheorem{definition}{\hei{定义}}
\newtheorem{axiom}{\hei{公理}}
\newtheorem{property}{\hei{性质}}
\newtheorem{proposition}{\hei{命题}}
\newtheorem{lemma}{\hei{引理}}
\newtheorem{corollary}{\hei{推论}}
\newtheorem{remark}{\hei{注解}}
\newtheorem{condition}{\hei{条件}}
\newtheorem{conclusion}{\hei{结论}}
\newtheorem{assumption}{\hei{假设}}

%重定义一些环境的名字
\renewcommand{\proofname}{\hei{证明}}
\renewcommand\tablename{\hei{表}}

%---SCRIPT-----------------------------------------------------------------------------------------
\newcommand{\cA}{\mathcal{A}}
\newcommand{\cB}{\mathcal{B}}
\newcommand{\cC}{\mathcal{C}}
\newcommand{\cD}{\mathcal{D}}
\newcommand{\cE}{\mathcal{E}}
\newcommand{\ce}{\mathcal{e}}
\newcommand{\cF}{\mathcal{F}}
\newcommand{\cG}{\mathcal{G}}
\newcommand{\cg}{\mathcal{g}}
\newcommand{\cH}{\mathcal{H}}
\newcommand{\cI}{\mathcal{I}}
\newcommand{\cJ}{\mathcal{J}}
\newcommand{\cK}{\mathcal{K}}
\newcommand{\cL}{\mathcal{L}}
\newcommand{\cM}{\mathcal{M}}
\newcommand{\cN}{\mathcal{N}}
\newcommand{\cO}{\mathcal{O}}
\newcommand{\cP}{\mathcal{P}}
\newcommand{\cQ}{\mathcal{Q}}
\newcommand{\cR}{\mathcal{R}}
\newcommand{\cS}{\mathcal{S}}
\newcommand{\cT}{\mathcal{T}}
\newcommand{\cU}{\mathcal{U}}
\newcommand{\cV}{\mathcal{V}}
\newcommand{\cW}{\mathcal{W}}
\newcommand{\cX}{\mathcal{X}}
\newcommand{\cY}{\mathcal{Y}}
\newcommand{\cZ}{\mathcal{Z}}
\newcommand{\cz}{\mathcal{z}}
%---BLACKBOARD-------------------------------------------------------------------------------------
\newcommand{\mA}{\mathbb A}
\newcommand{\mB}{\mathbb B}
\newcommand{\mC}{\mathbb C}
\newcommand{\mD}{\mathbb D}
\newcommand{\mE}{\mathbb E}
\newcommand{\mF}{\mathbb F}
\newcommand{\mG}{\mathbb G}
\newcommand{\mg}{\mathbb g}
\newcommand{\mH}{\mathbb H}
\newcommand{\mI}{\mathbb I}
\newcommand{\mJ}{\mathbb J}
\newcommand{\mK}{\mathbb K}
\newcommand{\mL}{\mathbb L}
\newcommand{\mM}{\mathbb M}
\newcommand{\mN}{\mathbb N}
\newcommand{\mO}{\mathbb O}
\newcommand{\mP}{\mathbb P}
\newcommand{\mQ}{\mathbb Q}
\newcommand{\mR}{\mathbb R}
\newcommand{\mS}{\mathbb S}
\newcommand{\mT}{\mathbb T}
\newcommand{\mU}{\mathbb U}
\newcommand{\mV}{\mathbb V}
\newcommand{\mW}{\mathbb W}
\newcommand{\mX}{\mathbb X}
\newcommand{\mY}{\mathbb Y}
\newcommand{\mZ}{\mathbb Z}
\newcommand{\mz}{\mathbb z}

\newcommand{\bV}{\mathbf{V}}
\newcommand{\bz}{\mathbf{z}}
\newcommand{\bT}{\mathbf{T}}
\newcommand{\bx}{\mathbf{x}}
\newcommand{\be}{\mathbf{e}}
\newcommand{\bff}{\mathbf{f}}
\newcommand{\bg}{\mathbf{g}}
\newcommand{\bn}{\mathbf{n}}
\newcommand{\bt}{\mathbf{t}}
\newcommand{\bd}{\mathbf{d}}
\newcommand{\bzero}{\mathbf{0}}
\newcommand{\bka}{\mathbf{\kappa}}

\newcommand{\rd}{\mathrm{d}}
%---SHORTCUTS--------------------------------------------------------------------------------------
\newcommand\xor{\mathbin{\char`\^}}
\DeclareMathOperator{\res}{Res}
\DeclareMathOperator{\sgn}{sgn}
\DeclareMathOperator{\supp}{supp}
\DeclareMathOperator{\as}{as}
\newcommand{\slant}[1]{\slshape #1\normalfont}
\newcommand{\dd}[2]{\frac{d#1}{d#2}} 
\newcommand{\ddx}{\frac{d}{dx}}
\newcommand{\ddt}{\frac{d}{dt}}
\newcommand{\dds}{\frac{d}{ds}}
\newcommand{\pd}[1]{\ds\frac{\partial}{\partial #1 }}
\newcommand{\pdd}[2]{\ds\frac{\partial #1}{\partial #2 }}
\newcommand{\mdd}[3]{\ds\frac{\partial^{#3} #1}{\partial #2^{#3} }}
\newcommand{\x}{\ _\Box}
\newcommand{\ds}{\displaystyle}
\newcommand{\bs}{\backslash}
\newcommand{\Bold}{\noindent \bfseries}
\newcommand{\Norm}{\normalfont}
\newcommand{\exl}[1]{\textcolor{NavyBlue}{\Bold Exercise #1 \Norm}}
\newcommand{\ex}{\textcolor{NavyBlue}{\Bold Problem: \Norm}}
\newcommand{\sol}{\textcolor{Mulberry}{\Bold Solution: \Norm}}
\newcommand{\pf}{\textcolor{Mulberry}{\Bold Proof: \Norm}}
\newcommand{\Title}[1]{\LARGE\Bold \textcolor{Sepia}{#1}\Norm\normalsize \vspace{10pt} \newline}
\newcommand{\prop}{\Bold \textcolor{YellowOrange}{ Proposition:} \Norm}
\newcommand{\propl}[1]{\Bold \textcolor{YellowOrange}{ Proposition #1:} \Norm}
\newcommand{\rk}{\Bold \textcolor{YellowOrange}{ Remark:} \Norm}
\newcommand{\rmk}[1]{\Bold\textcolor{YellowOrange}{#1} \Norm}
\newcommand{\thm}[1]{\Bold \textcolor{YellowOrange}{ Theorem #1} \Norm}
\newcommand{\ind}{\indent\indent}
\newcommand{\br}{\vspace{10pt} \newline}

\newcommand{\red}{\color{red}}
\newcommand{\blue}{\color{blue}}



\begin{document}

\title[数值线性代数]{{\small\leftline{数值线性代数———}~~~~~~~~~~~~~~~~~~~~~~~~~~~~~~~~~~~~~~~~~~~~~~
~~~~~~~~~~~} \\
矩阵计算
}




\author[]{~~易年余~~}

\institute[湘潭大学数学系]

\date[\today]

%\pgfdeclareimage[height=1cm]{institution-logo}{figures/xtu.pdf}

%\logo{\pgfuseimage{institution-logo}}


\frame[plain]{\titlepage}


\AtBeginSection[]{

  \frame<beamer>{ 

    \frametitle{线性代数基础}   

    \tableofcontents[currentsection] 

  }
}
\section{线性空间与内积空间}
\begin{frame}
\frametitle{计算数学的主要任务}
$\bullet$数域, 如: Q, R, C

$\bullet$线性空间, 如: $R^n,C^n,R^{m \times n}$

$\bullet$线性相关与线性无关, 秩, 基, 维数

$\bullet$线性子空间

$\bullet$像空间 (列空间, 值域) Ran(A), 零空间 (核) Ker(A)

$\bullet$张成子空间:$span\{x_1, x_2, . . . , x_k\}, span(A) = Ran(A)$
\end{frame}
\begin{frame}
\frametitle{直和}
设$S_1, S_2$是子空间, 若$S_1 + S_2$中的任一元素都可唯一表示成

$$x = x_1 + x_2, x_1 \in S1, x2 \in S2,$$

则称$S_1 + S_2$为直和, 记为$S_1 \oplus S_2$.


\begin{theorem}
	
	设 $S_1$ 是 $S$ 的子空间, 则存在另一个子空间 $S_2$, 使得$$S = S_1 \oplus S_2.$$
	
\end{theorem}

{\bfseries  例:} \quad 设$A \in \mathbb{C}^{m \times n}$, 则
$$\mathbb{C}^{n}=\operatorname{Ker}(A) \oplus \operatorname{Ran}\left(A^{*}\right), \quad \mathbb{C}^{m}=\operatorname{Ker}\left(A^{*}\right) \oplus \operatorname{Ran}(A)$$ 
\end{frame}

\begin{frame}
\frametitle{内积空间}
\qquad$\bullet$内积, 内积空间, 欧氏空间, 酉空间
 
\qquad$\bullet$常见内积空间:
 
 \qquad\qquad - $C^{n} : (x, y) = y∗x$
 
 \qquad\qquad - $R^{n} : (x, y) = y^{T}x$
 
 \qquad\qquad - $R^{m \times n} : (A, B) = tr(B^{T}A)$
 
 正交与正交补:
 
\qquad$\bullet$正交: 向量正交, 子空间正交
 
\qquad$\bullet$正交补空间
\end{frame}

\section{向量范数与矩阵范数}
\begin{frame}
\frametitle{向量范数与矩阵范数}
\begin{definition}(向量范数)
	若函数$f : C^n \to R $满足
	
	(1) $f(x) \geq 0, ∀ x \in C^n$, 等号当且仅当 $x = 0$ 时成立;
	
	(2) $f(\alpha x) = |\alpha| · f(x), ∀ x \in C^{n}, α \in C;$
	
	(3) $f(x + y) \leq f(x) + f(y), ∀x, y \in C^{n};$
	
	则称 f(x) 为$C^{n}$上的范数, 通常记作 $|| \cdot ||$
\end{definition}

相类似地,我们可以定义实数空间$R^n$上的向量范数。
\end{frame}

\begin{frame}

\frametitle{常见的向量范数}
$\bullet$ 1-范数:$||x||_1=|x_1|+|x_2|+...+|x_n|$

$\bullet$ 2-范数:$||x||_2=\sqrt{|x_1|^2+|x_2|^2+...+|x_n|^2}$

$\bullet$ ∞-范数:$\|x\|_{\infty}=\max _{1 \leq i \leq n}\left|x_{i}\right|$

$\bullet$ p-范数:$\|x\|_{p}=\left(\sum_{i=1}^{n}\left|x_{i}\right|^{p}\right)^{1 / p}, \quad 1 \leq p<\infty$
\end{frame}

\begin{frame}
\frametitle{常见的向量范数}
\begin{definition}(范数等价性)
	$\mathbb{C}^{n}$上的向量范数$||\cdot||_{\alpha}$与$||\cdot||_{\beta}$等价: 存在正常数$c_1, c_2,$使得
	$$c_{1}\|x\|_{\alpha} \leq\|x\|_{\beta} \leq c_{2}\|x\|_{\alpha}, \quad \forall x \in \mathbb{C}^{n}$$
\end{definition}

\begin{theorem}
	$\mathbb{C}^{n}$空间上的所有向量范数都是等价的, 特别地, 有
	$$
	\|x\|_{2} \leq\|x\|_{1} \leq \sqrt{n}\|x\|_{2}
	$$
	$$
	\|x\|_{\infty} \leq\|x\|_{2} \leq \sqrt{n}\|x\|_{\infty}
	$$
	$$
	\|x\|_{\infty} \leq\|x\|_{1} \leq n\|x\|_{\infty}
	$$
\end{theorem}
\end{frame}

\begin{frame}
\frametitle{常见的向量范数}
\begin{theorem}(Cauchy-Schwartz不等式)
	设$(.,.)$是$\mathbb{C}^{n}$上的内积,则对任意$x,y \in \mathbb{C}^{n}$,有
	$$
	|(x, y)|^{2} \leq(x, x) \cdot(y, y)
	$$
\end{theorem}

\begin{corollary}
	设$(.,.)$是$\mathbb{C}^{n}$上的内积,则$||x||\triangleq \sqrt{(x,x)}$是$\mathbb{C}^{n}$上的一个向量范数
\end{corollary}

\begin{theorem}
	设$||\cdot||$是$\mathbb{C}^{n}$上的一个向量范数,则$f(x)\triangleq||x||$是$\mathbb{C}^{n}$上的连续函数。
\end{theorem}
\end{frame}

\begin{frame}
\frametitle{矩阵范数}
\begin{definition}(矩阵范数)
	若函数$f : \mathbb{C}^{n \times n} \to R$ 满足
	
	$(1) f(A) \geq 0, \forall  A \in \mathbb{C}^{n \times n},$ 等号当且仅当 $A = 0$时成立;
	
	$(2) f(\alpha A) = |\alpha| · f(A), ∀ A \in \mathbb{C}^{n\times n}, α \in \mathbb{C};$
	
	$(3) f(A + B) \leq f(A) + f(B), \forall A, B \in \mathbb{C}^{n \times n};$
	则称 $f(x) $为$\mathbb{C}
	^{n \times n} $上的范数, 通常记作 $|| \cdot ||$。
\end{definition}

相容的矩阵范数:$f(AB) \leq f(A)f(B), \forall A, B \in  \mathbb{C}^{n \times n}$。

若未明确指出, 讲义所涉及矩阵范数都指相容矩阵范数
\end{frame}

\begin{frame}
\frametitle{矩阵范数}
\begin{lemma}
	设 $∥ · ∥ $是 $C^{n}$上的向量范数, 则
	$$
	\|A\| \triangleq \sup _{x \in \mathbb{C}^{n}, x \neq 0} \frac{\|A x\|}{\|x\|}=\max _{\|x\|=1}\|A x\|
	$$
	是$\mathbb{C}
	^{n \times n} $上的范数, 称为算子范数, 或诱导范数, 导出范数。
\end{lemma}

 算子范数都是相容的, 且$$ \|A x\| \leq\|A\| \cdot\|x\|, \quad A \in \mathbb{C}^{n \times n}, x \in \mathbb{C}^{n}$$

 类似地, 我们可以定义$\mathbb{C}^{m \times n}, \mathbb{R}^{n \times n}, \mathbb{R}^{m \times n}$上的矩阵范数.
\end{frame}

\begin{frame}
\frametitle{矩阵范数}
\begin{lemma}
	可以证明:
	
	(1) 1-范数 (列范数):$\|A\|_{1}=\max _{1 \leq j \leq n}\left(\sum_{i=1}^{n}\left|a_{i j}\right|\right)$
	
	(2) ∞-范数 (行范数): $\|A\|_{\infty}=\max _{1 \leq i \leq n}\left(\sum_{j=1}^{n}\left|a_{i j}\right|\right)$
	
	(3) 2-范数 (谱范数):$\|A\|_{2}=\sqrt{\rho\left(A^{\top} A\right)}$
	
\end{lemma}

另一个常用范数 F-范数 $\|A\|_{F}=\sqrt{\sum_{i=1}^{n} \sum_{j=1}^{n}\left|a_{i j}\right|^{2}}$
\end{frame}

\begin{frame}
\frametitle{矩阵范数}
\begin{theorem}(矩阵范数的等价性)
	$\mathbb{R}^{n×n}$空间上的所有范数都是等价的, 特别地, 有
	$$
	\frac{1}{\sqrt{n}}\|A\|_{2} \leq\|A\|_{1} \leq \sqrt{n}\|A\|_{2},
	$$
	$$
	\frac{1}{\sqrt{n}}\|A\|_{2} \leq\|A\|_{\infty} \leq \sqrt{n}\|A\|_{2},
	$$
	$$
	\frac{1}{n}\|A\|_{\infty} \leq\|A\|_{1} \leq n\|A\|_{\infty}
	,
	$$
	$$
	\frac{1}{\sqrt{n}}\|A\|_{1} \leq\|A\|_{F} \leq \sqrt{n}\|A\|_{2}
	,
	$$
\end{theorem}
\end{frame}

\begin{frame}
\frametitle{矩阵范数的一些性质}
$\bullet$对任意的算子范数$||.||$,有$||I|| = 1$

$\bullet$对任意的相容范数$||.||$,有$||I|| \leq 1$ 

$\bullet$F-范数是相容的, 但不是算子范数

$\bullet$$||.||_2$和$||.||_F$酉不变范数

$\bullet$ $\left\|A^{\top}\right\|_{2}=\|A\|_{2},\left\|A^{\top}\right\|_{1}=\|A\|_{\infty}$

$\bullet$若$A$是正规矩阵, 则$\|A\|_{2}=\rho(A)$
\end{frame}

\begin{frame}
\frametitle{向量序列的收敛}
设${x^{(k)}}^{\inf}_{
	k=1}$是$\mathbb{C}^n $中的一个向量序列, 如果存在 $x \in \mathbb{C}^n$, 使得
$$
\lim _{k \rightarrow \infty} x_{i}^{(k)}=x_{i}, \quad i=1,2, \ldots, n
$$
则称${x^{(k)}}$(按分量)收敛到 x, 记为$\lim _{k \rightarrow \infty} x^{(k)}=x$

\begin{theorem}(矩阵范数的等价性)
	设$||.||$是$\mathbb{C}^{n}$上的任意一个向量范数, 则$\lim _{k \rightarrow \infty} x^{(k)}=x$的充要条件是
	$$
	\lim _{k \rightarrow \infty}\left\|x^{(k)}-x\right\|=0
	$$
\end{theorem}
\end{frame}

\begin{frame}
\frametitle{收敛速度}
设点列${εk}_{k=1}^{\inf}$收敛, 且$\lim _{k=\infty} \varepsilon_{k}=0$. 若存在一个有界常数 $0 < c < ∞$, 使得

$\lim _{k \rightarrow \infty} \frac{\left|\varepsilon_{k+1}\right|}{\left|\varepsilon_{k}\right|^{p}}=c$

则称点列 ${ε_k} $是 $p$ 次 (渐进) 收敛的. 若$ 1 < p < 2$ 或$ p = 1 $且 $c = 0,$ 则称点列是超线性收敛的.

$†$ 类似地, 我们可以给出矩阵序列的收敛性和判别方法.
\end{frame}

\section{矩阵与投影}
\begin{frame}
\frametitle{特征值与特征向量}
$\bullet$特征多项式, 特征值, 特征向量, 左特征向量, 特征对

$\bullet$n 阶矩阵 A 的谱: $σ(A) ≜ {λ1, λ2, . . . , λn}$

$\bullet$代数重数和几何重数, 特征空间

$\bullet$最小多项式

$\bullet$可对角化, 特征值分解

$\bullet$可对角化的充要条件

$\bullet$特征值估计: Bendixson 定理, 圆盘定理
\end{frame}

\begin{frame}
\frametitle{Bendixson定理}
设$A \in \mathbb{C}^{n \times n}$,令$H=\frac{1}{2}\left(A+A^{*}\right), S=\frac{1}{2}\left(A-A^{*}\right)$.则有
$$
\begin{array}{l}{\lambda_{\min }(H) \leq \operatorname{Re}(\lambda(A)) \leq \lambda_{\max }(H)} \\ {\lambda_{\min }(i S) \leq \operatorname{Im}(\lambda(A)) \leq \lambda_{\max }(i S)}\end{array}
$$
其中$Re(.)$和$Im(.)$分别表示实部和虚部。


† 一个矩阵的特征值的实部的取值范围由其 Hermite 部分确定, 而虚
部则由其 Skew-Hermite 部分确定.
\end{frame}

\begin{frame}
\frametitle{Gerschgorin圆盘定理}
设$A=\left[a_{i j}\right] \in \mathbb{C}^{n \times n}$,定义集合
$$
\mathcal{D}_{i} \triangleq\left\{z \in \mathbb{C} :\left|z-a_{i i}\right| \leq \sum_{j=1, j \neq i}^{n}\left|a_{i j}\right|\right\}, \quad i=1,2, \ldots, n
$$
这就是A的n个Gerschgorin圆盘。


\begin{theorem}(Gerschgorin圆盘定理)
	设$A=\left[a_{i j}\right] \in \mathbb{C}^{n \times n}$. 则 A 的所有特征值都包含在 A 的 Gerschgorin 圆盘的并集中, 即$\sigma(A) \subset \bigcup_{i=1}^{n} \mathcal{D}_{i}$
\end{theorem}
\end{frame}

\begin{frame}
\frametitle{投影变换与投影矩阵}
设 $S = S_1 \oplus S_2$, 则 $S$ 中的任意向量 $x$ 都可唯一表示为

$$x = x_1 + x_2, x_1 \in S_1, x_2 \in S_2.$$

我们称 $x_1$ 为 $x$ 沿 $S_2$ 到 $S_1$ 上的投影, 记为 $x|_{S_1}$
.
设线性变换$ P : S \to S. $如果对任意 $x \in S$, 都有

$$P x = x|_{S_1}$$,

则称 $P$ 是从 $S$ 沿 $S_2$ 到 $S_1$ 上的 投影变换 (或 投影算子), 对应的变换矩阵
称为 投影矩阵.
\end{frame}

\begin{frame}
\frametitle{投影变换与投影矩阵}
\begin{lemma}
	设 $P \in \mathbb{R}^{n×n} $是一个投影矩阵, 则
	\begin{equation}
	\mathbb{R}^{n}=\operatorname{Ran}(P) \oplus \operatorname{Ker}(P)
	\end{equation}
	反之, 若 $(1.3)$ 成立, 则 $P$ 是沿 $Ker(P)$ 到$ Ran(P)$ 上的投影
\end{lemma}

投影矩阵由其像空间和零空间唯一确定.


\begin{lemma}
	若 $S_1$ 和 $S_2$ 是 $\mathbb{R}^n$的两个子空间, 且$ \mathbb{R}^n= S_1 ⊕ S_2$, 则存在唯一的
	投影矩阵 $P$, 使得
	$$
	\operatorname{Ran}(P)=\mathcal{S}_{1}, \quad \operatorname{Ker}(P)=\mathcal{S}_{2}
	$$
\end{lemma}
\end{frame}

\begin{frame}
\frametitle{投影矩阵的判别}
\begin{theorem}
	矩阵 $P \in\mathbb{R}^{n \times n}$是投影矩阵的充要条件是 $P^2=P$
\end{theorem}

\end{frame}

\begin{frame}
\frametitle{投影算子的矩阵表示}
设 $S_1$ 和 $S_2$ 是 $\mathbb{R}^n$ 的两个 $m$ 维子空间. 如果$S_{1} \oplus \mathcal{S}_{2}^{\perp}=\mathbb{R}^{n}$ ,则存在唯一的
投影矩阵 P, 使得
$$
\operatorname{Ran}(P)=\mathcal{S}_{1}, \quad \operatorname{Ker}(P)=\mathcal{S}_{2}^{\perp}
$$
此时, 我们称 P 是 S1 上与 S2 正交的投影矩阵, 且有
$$
P=V\left(W^{\top} V\right)^{-1} W^{\top}
$$
其中 $V = [v_1, v_2, . . . , v_m] $和 $W = [w_1, w_2, . . . , w_m]$ 的列向量组分别构成
$S_1 $和 $S_2$ 的一组基.
\end{frame}

\begin{frame}
\frametitle{正交投影}
设 $S_1$ 是内积空间 $S $的一个子空间, $x \in S$, 则$x$ 可唯一分解成
$$x = x_1 + x_2, x_1 \in S_1, x_2 \in S^{⊥}_1$$
,
其中 $x_1$ 称为 $x$ 在 $S_1$ 上的正交投影.

\qquad$\bullet$若 $P$ 是沿 $S^{⊥}_1 $到 $S_1$ 上的投影变换, 则称 $P$ 为 $S_1$ 上的正交投影变换
(对应的矩阵为 正交投影矩阵), 记为 $P_{S_1}$

\qquad$\bullet$如果 $P $不是正交投影变换, 则称其为斜投影变换

\begin{theorem}
	投影矩阵 $P \in\mathbb{R}^{n×n}$是正交投影矩阵的充要条件 $P^{} = P$.
\end{theorem}
\end{frame}

\begin{frame}
\frametitle{正交投影}

\begin{corollary}
	设 $P$ 是子空间 $S1$ 上的 正交投影变换. 令 $v_1, v_2, . . . , v_m $是$ S_1$ 的
	一组标准正交基, 则
	$$
	P=V V^{\top}
	$$
	其中$ V = [v_1, v_2, . . . , v_m].$
\end{corollary}
\begin{property}
	设 P $\in$ $\mathbb{R}^{n×n}$ 是一个正交投影矩阵, 则
	$$
	||P||_{2}=1
	$$
	且对 $∀ x \in \mathbb{R}^n$, 有
	$$
	||x||_{2}^{2}=||P x||_{2}^{2}+||(I-P) x||_{2}^{2}
	$$
\end{property}
\end{frame}

\begin{frame}
\frametitle{正交投影矩阵的一个重要应用}

\begin{theorem}
	设 $S_1$ 是 $R_n$ 的一个子空间,$ z \in R_n$ 是一个向量. 则最佳逼近问题
	$$
	\min _{x \in \mathcal{S}_{1}}\|x-z\|_{2}
	$$
	的唯一解为
	$$
	x_{*}=P_{\mathcal{S}_{1}} z
	$$
	即 $S_1$ 中距离 $z$ 最近 (2-范数意义下) 的向量是 $z $在 $S
	_1$ 上的正交投影.
\end{theorem}
\end{frame}

\begin{frame}
\frametitle{正交投影矩阵的一个重要应用}

\begin{corollary}
	设矩阵 $A \in \mathbb{R}^{n×n}$ 对称正定, 向量 $x∗ \in S_1 ⊆ R_n.$ 则$ x_∗$ 是最佳逼近问题
	$$
	\min _{x \in \mathcal{S}_{1}}\|x-z\|_{A}
	$$
	的解的充要条件是
	$$
	A\left(x_{*}-z\right) \perp \mathcal{S}_{1}
	$$
	这里$\|x-z\|_{A} \triangleq\left\|A^{\frac{1}{2}}(x-z)\right\|_{2}$
\end{corollary}
\end{frame}

\begin{frame}
\frametitle{不变子空间}

设 $A \in \mathbb{R}^{n×n}$, $S$ 是 $\mathbb{R}^n$ 的一个子空间, 记
$$
A \mathcal{S} \triangleq\{A x : x \in \mathcal{S}\}
$$

\begin{definition}
	若$ AS ⊆ S$, 则称 $S$ 为 $A$ 的一个不变子空间.	
\end{definition}



\begin{theorem}
	设 $x_1, x_2, . . . , x_m $是 $A$ 的一组线性无关特征向量, 则
	$$
	\operatorname{span}\left\{x_{1}, x_{2}, \ldots, x_{m}\right\}
	$$
	是 $A$ 的一个 $m $维不变子空间.
\end{theorem}
\end{frame}

\begin{frame}
\frametitle{不变子空间的一个重要性质}
\begin{theorem}
	设 $A \in \mathbb{R}^{n \times n}, X \in \mathbb{R}^{n \times k}$ 且 $rank(X) = k$. 则 $span(X)$ 是 $A$ 的不变子空间的充要条件是存在 $B \in \mathbb{R}^{k \times k}$ 使得
	$$AX = XB,$$
	此时, $B$ 的特征值都是 $A$ 的特征值. 
\end{theorem}



\begin{corollary}
	设 $A \in \mathbb{R}^{n \times n}, X \in \mathbb{R}^{n\times k}$ 且 $rank(X) = k$. 若存在一个矩阵$B \in \mathbb{R}^{k\times k}$ 使得 
	
	$AX = XB$, 则 $(λ, v)$ 是 $B$ 的一个特征对当且仅当$(\lambda, Xv)$ 是 $A$ 的一个特征对.
\end{corollary}
\end{frame}

\section{矩阵标准型}
\begin{frame}
\frametitle{矩阵标准型}
计算矩阵特征值的一个基本思想是通过相似变换, 将其转化成一个形式
尽可能简单的矩阵, 使得其特征值更易于计算. 其中两个非常有用的特殊
矩阵是 $Jordan$ 标准型和 $Schur$ 标准型.
\end{frame}

\begin{frame}
\frametitle{矩阵标准型}
\begin{theorem}
	设 $A \in \mathbb{C}^{n×n}$ 有 $p$ 个不同特征值, 则存在非奇异矩阵$ X \in \mathbb{C}^{n×n}$, 使得
	$$
	X^{-1} A X=\left[\begin{array}{cccc}{J_{1}} & {} & {} & {} \\ {} & {J_{2}} & {} \\ {} & {} & {\ddots} & {} \\ {} & {} & {} & {J_{p}}\end{array}\right] \triangleq J
	$$
	其中 $J_i$ 的维数等于$ λ_i $的代数重数, 且具有下面的结构
	
	\centering {$J_{i}=\left[\begin{array}{ccccc}{J_{i 1}} & {} & {} & {} \\ {} & {J_{i 2}} \\ {} & {} & {\ddots} & {} \\ {} & {} & {} & {J_{i \nu_{i}}}\end{array}\right]$ $J_{i k}=\left[\begin{array}{ccccc}{\lambda_{i}} & {1} & {} & {} \\ {} & {\ddots} & {\ddots} & {} \\ {} & {} & {\lambda_{i}} & {1} \\ {} & {} & {} & {\lambda_{i}}\end{array}\right]$}	
	
	这里 $ν_i$ 为 $λ_i$ 的几何重数, $J_{ik}$ 称为 $\blue{Jordan \text{块}}$, 每个 $Jordan$ 块对应一个特征向量	
\end{theorem}
\end{frame}

\begin{frame}
\frametitle{矩阵标准型}
$\red{\dagger}$ Jordan 标准型在理论研究中非常有用, 但数值计算比较困难, 目前还
没有找到十分稳定的数值算法.



\begin{corollary}
	所有可对角化矩阵组成的集合在所有矩阵组成的集合中是稠密
	的.
\end{corollary}
\end{frame}

\begin{frame}
\frametitle{Schur 标准型}
\begin{theorem}
	设 A $\in$ $\mathbb{C}^{n \times n}$, 则存在一个酉矩阵 $U \in \mathbb{C}^{n \times n}$ 使得	
	\centering {$U^{*} A U=\left[\begin{array}{cccc}{\lambda_{1}} & {r_{12}} & {\cdots} & {r_{1 n}} \\ {0} & {\lambda_{2}} & {\cdots} & {r_{2 n}} \\ {\vdots} & {} & {\ddots} & {\vdots} \\ {0} & {\cdots} & {0} & {\lambda_{n}}\end{array}\right] \triangleq R$ \text{或} $A=U R U^{*}$}
	
	其中 $\lambda_1$, $\lambda_2$, . . . , $\lambda_n$ 是 $A$ 的特征值 (排序任意).
\end{theorem}
\end{frame}

\begin{frame}
\frametitle{Schur 标准型}
关于 Schur 标准型的几点说明:

$\bullet$Schur 标准型可以说是酉相似变化下的最简形式

$\bullet$U 和 R 不唯一, R 的对角线元素可按任意顺序排列

$\bullet$A 是正规矩阵当且仅当 上述定理中的 R 是对角矩阵;

$\bullet$A 是 Hermite 矩阵当且仅当 上述定理 中的 R 是实对角矩阵.
\end{frame}

\begin{frame}
\frametitle{实 Schur 标准型}
\begin{theorem}
	设 $A \in \mathbb{R}^{n \times n}$, 则存在正交矩阵 $Q \in \mathbb{R}^{n \times n}$, 使得
	$$
	Q^{\top} A Q=T
	$$
	其中 $T \in \mathbb{R}^{n \times n}$是 $\blue\text{拟上三角矩阵}$, 即$ T$ 是块上三角的, 且对角块为 $1 × 1$
	或 $2 \times 2$ 的块矩阵. 若对角块是 $1 \times 1$ 的, 则其就是 $A$ 的一个特征值, 若
	对角块是 $2 \times 2 $的, 则其特征值是 $A$ 的一对共轭复特征值.
\end{theorem}
\end{frame}

\section{几类特殊矩阵}
\begin{frame}
\frametitle{对称正定矩阵}
设 $A \in \mathbb{C}^{n \times n}$.

A 是 半正定 $\Longleftrightarrow$ $\operatorname{Re}\left(x^{*} A x\right) \geq 0, \forall x \in \mathbb{C}^{n}$

A 是 正定 $\Longleftrightarrow$ $\operatorname{Re}\left(x^{*} A x\right)>0, \forall x \in \mathbb{C}^{n}, x \neq 0$

A 是 Hermite 半正定 $\Longleftrightarrow$ A Hermite 且半正定

A 是 Hermite 正定 $\Longleftrightarrow$ A Hermite 且正定

正定和半正定矩阵不要求是对称或 Hermite 的
\end{frame}

\begin{frame}
\frametitle{对称正定矩阵}
\begin{theorem}
	设 $A \in \mathbb{C}^{n\times n}$. 则 $A$ 正定 (半正定) 的充要条件是矩阵$ H =\frac{1}{2}(A +A∗)$ 正定 (半正定). 
\end{theorem}

\begin{theorem}
	设 $A \in \mathbb{R}^{n \times n}$. 则 $A$ 正定 (或半正定) 的充要条件是对任意非零向量 $x \in \mathbb{R}^n$ 有 $x^{\top}Ax > 0 $(或$ x^{\top}Ax \geq 0$). 
\end{theorem}
\end{frame}

\begin{frame}
\frametitle{矩阵平方根}
\begin{theorem}
	设 $A \in \mathbb{C}^{n \times n}$ 是 Hermite 半正定, $k $是正整数. 则存在唯一的Hermite 半正定矩阵 $B \in \mathbb{C}^{n \times n}$ 使得
	$$B^k = A.$$
	同时, 我们还有下面的性质:
	
	(1) $BA = AB$, 且存在一个多项式 $p(t)$ 使得$ B = p(A)$;
	
	(2) $rank(B) = rank(A)$, 因此, 若$ A$ 是正定的, 则 $B$ 也正定;
	
	(3) 如果 $A$ 是实矩阵的, 则 $B$ 也是实矩阵.
\end{theorem}


特别地, 当 k = 2 时, 称 B 为 A 的平方根, 通常记为 $A^{\frac{1}{2})}$.
\end{frame}

\begin{frame}
\frametitle{矩阵平方根}
Hermite 正定矩阵与内积之间有下面的关系

\begin{theorem}
	设 $(·, ·)$ 是 $\mathbb{C}^n$上的一个内积, 则存在一个 Hermite 正定矩阵 $A  \mathbb{C}^{n \times n}$使得
	$$(x, y) = y^{∗}Ax.$$
	反之, 若 $A \in  \mathbb{C}^{n \times n} $是 Hermite 正定矩阵, 则
	$$
	f(x, y) \triangleq y^{*} A x
	$$
	是 $\mathbb{C}^n $上的一个内积. 
\end{theorem}

 上述性质在实数域中也成立.
\end{frame}

\begin{frame}
\frametitle{对角占优矩阵}
\begin{definition}
	设$ A \in\mathbb{C}^{n \times n}$, 若
	$$
	\left|a_{i i}\right| \geq \sum_{j \neq i}\left|a_{i j}\right|
	$$
	对所有 $i = 1, 2, . . . , n $都成立, 且至少有一个不等式严格成立, 则称 A
	为弱行对角占优. 若对所有 $i = 1, 2, . . . , n$ 不等式都严格成立, 则称 A
	是严格行对角占优. 通常简称为弱对角占优和严格对角占优.	
\end{definition}

类似地, 可以定义弱列对角占优 和 严格列对角占优.
\end{frame}

\begin{frame}
\frametitle{可约与不可约}
设$A \in \mathbb{R}^{n \times n}$, 若存在置换矩阵 P, 使得$P A P^{\top}$为块上三角, 即
$$
P A P^{\top}=\left[\begin{array}{cc}{A_{11}} & {A_{12}} \\ {0} & {A_{22}}\end{array}\right]
$$
其中 $A_{11} \in \mathbb{R}^{k \times k}(1 \leq k<n)$, 则称 A 为可约, 否则不可约.

\begin{theorem}
	设 $A \in \mathbb{R}^{n \times n}$, 指标集$\mathbb{Z}_{n}=\{1,2, \ldots, \ldots, n\}$. 则 A 可约的充要条
	件是存在非空指标集 $J \subset \mathbb{Z}_{n}$ 且 $J \neq \mathbb{Z}_{n}$, 使得
	$$
	a_{i j}=0, \quad i \in J \text{且} j \in \mathbb{Z}_{n} \backslash J
	$$
	这里 $\mathbb{Z}_n \backslash J $表示 J 在$ \mathbb{Z}_n $中的补集.	
\end{theorem}
\end{frame}

\begin{frame}
\frametitle{可约与不可约}
\begin{theorem}
	若$A \in \mathbb{C}^{n \times n}$严格对角占优, 则 A 非奇异	
\end{theorem}

\begin{theorem}
	若$A \in \mathbb{C}^{n \times n}$不可约对角占优, 则 A 非奇异	
\end{theorem}
\end{frame}

\begin{frame}
\frametitle{其他常见特殊矩阵}
$\bullet$带状矩阵:
$a_{i j} \neq 0$ only if $-b_{u} \leq i-j \leq b_{l}$
, 其中 $b_u$ 和 $b_l$ 为非负整数, 分别称为下
带宽和上带宽, $b_u + b_l + 1 $称为 A 的带宽

$\bullet$上 Hessenberg 矩阵: $a_{ij} = 0\quad for\quad i - j > 1$,
$$
\left[\begin{array}{ccccc}{*} & {*} & {*} & {\dots} & {*} \\ {*} & {*} & {*} & {\dots} & {*} \\ { } & {*} & {*} & {\dots} & {*} \\ { } & { } & {\ddots} & {\ddots} & {\vdots}\\ { } & { } & { } & {*} & {*}\end{array}\right]
$$

$\bullet$下 Hessenberg 矩阵
\end{frame}

\begin{frame}
\frametitle{其他常见特殊矩阵}
$\bullet$Toeplitz 矩阵
$$
T=\left[\begin{array}{cccc}{t_{0}} & {t_{-1}} & {\dots} & {t_{-n+1}} \\ {t_{1}} & {\ddots} & {\ddots} & {\vdots} \\ {\vdots} & {\ddots} & {\ddots} & {t_{-1}} \\ {t_{n-1}} & {\cdots} & {t_{1}} & {t_{0}}\end{array}\right]
$$

$\bullet$循环矩阵 (circulant):
$$
C=\left[\begin{array}{ccccc}{c_{0}} & {c_{n-1}} & {c_{n-2}} & {\cdots} & {c_{1}} \\ {c_{1}} & {c_{0}} & {c_{n-1}} & {\cdots} & {c_{2}} \\ {c_{2}} & {c_{1}} & {c_{0}} & {\cdots} & {c_{3}} \\ {\vdots} & {\vdots} & {\vdots} & {\ddots} & {\vdots} \\ {c_{n-1}} & {c_{n-2}} & {c_{n-3}} & {\cdots} & {c_{0}}\end{array}\right]
$$
\end{frame}

\begin{frame}
\frametitle{其他常见特殊矩阵}
Hankel 矩阵:
$$
H=\left[\begin{array}{ccccc}{h_0} & {h_1} & {\cdots} & {h_{n-2}} & {h_{n-1}} \\ {h_1} & {\ddots} & {\ddots} & {\ddots} & {h_n}\\ {\vdots} & {\ddots} & {\ddots} & {\ddots} & {\vdots}  \\ {h_{n-2}} & {\cdots} & {\cdots} & {\cdots} & {h_{2n-2}} \\ {h_{n-1}} & {h_n} & {\dots} & {h_{2n-2}} & {h_{2n-1}}\end{array}\right]
$$
\end{frame}

\section{Kronecker 积}
\begin{frame}
\frametitle{Kronecker 积}
\begin{definition}
	设$A \in \mathbb{C}^{m \times n}, B \in \mathbb{C}^{p \times q}$, 则 A 与 B 的 Kronecker 积定义为
	$$
	A \otimes B=\left[\begin{array}{cccc}{a_{11} B} & {a_{12} B} & {\cdots} & {a_{1 n} B} \\ {a_{21} B} & {a_{22} B} & {\cdots} & {a_{2 n} B} \\ {\vdots} & {\vdots} & {\ddots} & {\vdots} \\ {a_{m 1} B} & {a_{m 2} B} & {\cdots} & {a_{m n} B}\end{array}\right] \in \mathbb{C}^{m p \times n q}
	$$
	
	Kronecker 积也称为直积, 或张量积.
\end{definition}


$\dagger$ 任意两个矩阵都存在 Kronecker 积, 且 $A \oplus B$ 和 $B \oplus A$ 是同阶矩阵,
但通常 $A \otimes B \neq B \otimes A$
\end{frame}

\begin{frame}
\frametitle{基本性质}
(1) $(αA) \oplus B = A \oplus (αB) = α(A \oplus B), ∀ α \in \mathbb{C}$

(2) $(A \oplus B)^{⊺} = A^{⊺} \oplus B^{⊺}, (A \oplus B)^{∗} = A^{∗} \oplus B^{∗}$

(3) $(A \oplus B) \oplus C = A \oplus (B \oplus C)$

(4) $(A + B) \oplus C = A \oplus C + B \oplus C$

(5) $A \oplus (B + C) = A \oplus B + A \oplus C$

(6)混合积:$ (A \oplus B)(C \oplus D) = (AC) \oplus (BD)$

(7) $(A_1 \oplus A_2 \oplus · · · \oplus A_k)(B_1 \oplus B_2 \oplus · · · \oplus B_k)
= (A_1B_1) \oplus (A_2B_2) \oplus · · · \oplus (A_kB_k)$

(8) $(A_1 \oplus B_1)(A_2 \oplus B_2)· · ·(A_k \oplus B_k)
= (A_1A_2 · · · A_k) \oplus (B_1B_2 · · · B_k)$

(9) $rank(A \oplus B) = rank(A)rank(B)$
\end{frame}

\begin{frame}
\frametitle{Kronecker 积}
\begin{theorem}
	设 $A \in \mathbb{C}^{m \times m}, B \in \mathbb{C}^{n \times n},$ 并设 $(\lambda, x)$ 和 $(\mu, y)$ 分别是$ A $和$ B $的一个特征对, 则 $(\lambda \mu, x \oplus y) $是 $A \oplus B$ 的一个特征对. 由此可知,$ B \oplus A $与$A \oplus B$ 具有相同的特征值.	
\end{theorem}
\begin{theorem}
	设 $A \in \mathbb{C}^{m \times m}$, $B \in \mathbb{C}^{n \times n}$, 则
	
	(1) $tr(A \oplus B) = tr(A)tr(B)$ ;
	
	(2) $det(A \oplus B) = det(A)^n det(B)^m $;
	
	(3) $A \oplus I_n + I_m \oplus B $的特征值为 $\lambda_i + \mu_j$ , 其中 $\lambda_i$ 和 $\mu_j$ 分别为 A 和 B 的特征值;
	
	(4) 若 A 和 B 都非奇异, 则 $(A \oplus B)^{-1} = A^{-1} \oplus B^{-1}$;	
\end{theorem}
\end{frame}

\begin{frame}
\frametitle{Kronecker 积}
\begin{corollary}
	设$A=Q_{1} \Lambda_{1} Q_{1}^{-1}, B=Q_{2} \Lambda_{2} Q_{2}^{-1}$,则
	$$
	A \otimes B=\left(Q_{1} \otimes Q_{2}\right)\left(\Lambda_{1} \otimes \Lambda_{2}\right)\left(Q_{1} \otimes Q_{2}\right)^{-1}
	$$	
\end{corollary}

\begin{theorem}
	设$A \in \mathbb{C}^{m \times m}, B \in \mathbb{C}^{n \times n}$,则存在 m + n 阶置换矩阵 P 使得
	
	$$
	P^{\top}(A \otimes B) P=B \otimes A
	$$	
\end{theorem}
\end{frame}

\begin{frame}
\frametitle{Kronecker 积}
\begin{theorem}
	设矩阵 $X = [x_1, x_2, . . . , x_n] \in \mathbb{R}^{
		m \times n}$, 记 $vec(X)$为 $X$ 按列拉成的
	mn 维列向量, 即
	$$
	\operatorname{vec}(X)=\left[x_{1}^{\top}, x_{2}^{\top}, \ldots, x_{N}^{\top}\right]^{\top}
	$$
	则有
	$$vec(AX) = (I \oplus A)vec(X), vec(XB) = (B^{\top}\oplus I)vec(X),$$
	以及
	$$(A \oplus B)vec(X) = vec(BXA^{\top}
	)$$
\end{theorem}
\end{frame}
\end{document}
